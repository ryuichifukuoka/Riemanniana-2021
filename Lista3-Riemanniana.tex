\documentclass[twoside,openright,titlepage,numbers=noenddot,headinclude,  lineheaders footinclude=true,cleardoublepage=empty,BCOR=5mm,paper=a4,fontsize=12pt ]{scrbook} 
\usepackage[usenames,dvipsnames,svgnames,table]{xcolor}
\usepackage[unicode]{hyperref}
\usepackage[utf8]{inputenc}
\usepackage[eulerchapternumbers,beramono]{classicthesis} 
\usepackage{graphicx}
\usepackage[brazil,portuguese, english]{babel}
\usepackage{hyperref}
\usepackage{mathtools}
\usepackage{relsize}
\usepackage{amsmath,pict2e}
\usepackage{amsfonts}
\usepackage{amssymb}
\usepackage{amsthm}
\usepackage{breqn}
\usepackage{tikz-cd}
\usepackage{pst-node}
\usepackage{pdfpages}
\usepackage{makeidx}
\usepackage{blindtext}
\usepackage[intoc, portuguese]{nomencl}
\usepackage{nomencl}
\usepackage[T1]{fontenc}


\addtolength{\headsep}{0.6cm}
\setlength{\textheight}{22.7cm}
\setlength{\textwidth}{16.2cm}
\setlength{\oddsidemargin}{0cm}
\setlength{\evensidemargin}{0cm}

\linespread{1.3}

\newtheorem{teo}{Teorema}[chapter]
\newtheorem{lema}[teo]{Lema}
\newtheorem{prop}[teo]{Proposição}
\newtheorem{afirm}[teo]{Afirmação}
\newtheorem{cor}[teo]{Corolário}
\newtheorem{defin}[teo]{Definição}
\newtheorem{obs}[teo]{Observação}
\newtheorem{exemplo}[teo]{Exemplo}
\newcommand*{\QEDA}{\hfill\ensuremath{\blacksquare}}
\DeclareMathOperator{\sign}{sgn}
\newcommand{\R}{\mathbb R}
\newcommand{\N}{\mathbb N}
\newcommand{\RN}{\mathbb R^n}
\newcommand{\RM}{\mathbb R^m}
\newcommand{\Q}{\mathbb Q}
\newcommand{\Z}{\mathbb Z}
\newcommand{\K}{\mathbb K}
\newcommand{\mc}{\mathcal}
\newcommand{\ds}{\displaystyle}
\newcommand{\cqd}{\begin{flushright}$_\blacksquare$\end{flushright}}
\newcommand{\ck}{\mc{C}^k}
\newcommand{\g}{\mathfrak g}
\newcommand{\h}{\mathfrak h}
\newcommand{\gl}{\mathfrak {gl}}
\newcommand{\Ad}{\hbox{Ad}}
\newcommand{\ad}{\hbox{ad}}
\newcommand{\e}{\mathfrak e}
\newcommand{\uu}{\mathfrak{u}}
\makeatletter
\newcommand{\bigcomp}{%
  \DOTSB
  \mathop{\vphantom{\sum}\mathpalette\bigcomp@\relax}%
  \slimits@
}
\newcommand{\bigcomp@}[2]{%
  \begingroup\m@th
  \sbox\z@{$#1\sum$}%
  \setlength{\unitlength}{0.9\dimexpr\ht\z@+\dp\z@}%
  \vcenter{\hbox{%
    \begin{picture}(1,1)
    \bigcomp@linethickness{#1}
    \put(0.5,0.5){\circle{1}}
    \end{picture}%
  }}%
  \endgroup
}
\newcommand{\bigcomp@linethickness}[1]{%
  \linethickness{%
      \ifx#1\displaystyle 2\fontdimen8\textfont\else
      \ifx#1\textstyle 1.65\fontdimen8\textfont\else
      \ifx#1\scriptstyle 1.65\fontdimen8\scriptfont\else
      1.65\fontdimen8\scriptscriptfont\fi\fi\fi 3
  }%
}

\makeatother
\title{Lista 3 Riemanniana}
\author{}
\date{September 2021}

\begin{document}


\maketitle


\textbf{Exercício 01:} 
Considere $0 < r < s$. 
Sejam $B(0,r), B(0,s)\subset \mathbb{R}^n$ as bolas euclidianas abertas centradas na origem com raios $r$ e $s$, respectivamente. 
Mostre que existe uma função diferenciável $f : \mathbb{R}^n \rightarrow \mathbb{R}$ tal que $f(x) = 1$ se $x \in \overline{B}(0,r)$ e $ f(x) = 0 $ se $ x \notin B(0,s)$.
\vspace{0.5cm}

\textbf{Resolução:} 
Seja $F: \mathbb{R} \rightarrow \mathbb{R} $ a aplicação definida por
\[
    F(t) = \begin{cases}
                \text{exp}{\left(-\frac{1}{t}\right)} & \text{se } t > 0 \\
                0  & \text{se } t \leq 0
            \end{cases}
\]
Note que $ F $ é diferenciável em qualquer valor não nulo. 
Além disso, observe que $F$ é contínua, visto que $\lim _{t \to 0} \exp{\left(-\frac{1}{t}\right)} = 0$. 
Veja que para todo $k \in \mathbb{N}$, temos
\[
    \lim _{t^+ \to 0}\dfrac{\exp\left( -\frac{1}{t}\right)}{t^k} = \lim_{t^+ \to 0} \dfrac{t^{-k}}{\exp\left( \frac{1}{t}\right)}.
\]
Provemos por indução em $k$ que
\begin{equation}\label{equation:eq1}
    \lim _{t^+ \to 0}\dfrac{t^{-k}}{\exp\left( \frac{1}{t}\right)} = 0
\end{equation}
Se $k=0$ a condição é obvia. 
Suponha que seja verificada para algum $k >0$. Neste caso,
\[
    \lim _{t^+ \to 0} \dfrac{t^{- (k+1)}}{\exp\left( \frac{1}{t}\right)} 
    = \lim_{t^+ \to 0}   \dfrac{t^{- k - 1 }}{\exp\left( \frac{1}{t}\right)} 
    = \lim _{t^+ \to 0}  \dfrac{(-k-1)t^{- k - 2 }}{\frac{-1}{t^2}\exp\left( \frac{1}{t}\right)}
    =(k+1)\lim _{t^+ \to 0} \dfrac{t^{- k }}{\exp\left( \frac{1}{t}\right)}
    =0.
\]
Vejamos que para $t > 0 $
\[
    \frac{d^k}{d^k t} F(t) 
    = p_k (t) \dfrac{\exp\left(  -\frac{1}{t} \right)}{t^{2k}},  
\]
para algum polinômio $p_k$. 
É claramente válido para $ k = 0 $, tomando $p_0(t) = 1$. Suponha válido para algum $ k \in \mathbb{N}$. 
Note que, para $t>0$,
\[
    \begin{array}{rcl}
        \frac{d^{k+1}}{d^{k+1} t} F (t) &=& \frac{d}{d t} \left(p_k (t) \dfrac{\exp\left(  -\frac{1}{t} \right)}{t^{2k}}  \right) \\
        &=& \dfrac{p'_k (t)}{t^{2k}} \exp\left(  -\frac{1}{t} \right)  - 2k.\dfrac{p _ k(t)}{ t^{2k + 1} }\exp\left(  -\frac{1}{t} \right)  + \dfrac{p _ k(t)}{ t^{2k + 2} }\exp\left(  -\frac{1}{t} \right) \\
        &=& (t^2 p'_ k (t) - 2kt p _k(t) + p_k (t)) \dfrac{\exp\left(  -\frac{1}{t} \right)}{t^{2(k+1)}}\\
        &=& p _{k+1} (t) \dfrac{\exp\left(  -\frac{1}{t} \right)}{t^{2(k+1)}}.
    \end{array}
\]
Como $p _k (t)$ é um polinômio e \eqref{equation:eq1}, segue que
\[
    \lim _{t^+ \to 0} \frac{d^{k+1}}{d^{k+1} t} F (t) = 0,
\]
Mostremos que a $k$-ésima derivada à direita de $F$ em $t = 0$ coincide com a derivada à esquerda que é zero. 
De fato,
\[
    \frac{d^{k+1}}{d^{k+1} t} F (0) 
    = \lim_{t \rightarrow 0^+}  \dfrac{p _k (t) \dfrac{\exp\left(  -\frac{1}{t} \right)}{t^{2k} } - 0}{t} 
    =  \lim_{t \rightarrow 0^+}  p _{k} (t) \dfrac{\exp\left(  -\frac{1}{t} \right)}{t^{2k+1}}  = 0.
\]
Portanto, $f$ é diferenciável. 
Denote $\alpha  = \frac{s+r}{2}$. 
Considere $h: \mathbb{R} \rightarrow [0,1]$ definida por
\[
    h(t) =  \dfrac{F(\alpha -t)}{F(\alpha - t ) + F(t-r)}.
\]
A função $h$ está bem definida. 
De fato, inicialmente que $F(\alpha  -   t ) + F(t - r) >0$. Sabemos que $F(t)\geq 0$, para todo $t \in \mathbb{R}$ e $F(t)  = 0 $ se, e somente se, $t \leq 0$. 
Como $\alpha  - t \leq 0$ ocorre se, e somente se, $ t\leq  \alpha  = \frac{r+s}{2} > r$, ou seja, $ t-r > 0$, segue o afirmado. Além disso,
\[
    0 \leq h(t)   = \dfrac{F(\alpha -t)}{F(\alpha - t ) + F(t-r)} \leq  \dfrac{F(\alpha -t)}{F(\alpha - t ) } 
    = 1,
\]
o que prova que está bem definido o contradomínio. 
Portanto, $h$ é diferenciável, $h(t) = 0$ se $t \geq \alpha $ e $h (t) = 1$, se $t \leq r $, pois neste caso $ f(t-r) = 0$. 
Definindo $f: \mathbb{R}^n\rightarrow \mathbb{R}$ dado por $f(x) = h(|x|)$. 
Como $h$ é constante igual a $1$ em $x \in \overline{B}(0, r) $, segue que $f$ é diferenciável na origem e a diferenciabilidade nos demais pontos é imediata. 
Do que provamos para $h$ se $|x|\geq \alpha$, então $f(x) = h(|x|) = 0$. 
Em particular, se $x \notin \overline{B} (0, s)$, temos $f(x) =0$, concluindo o resultado. 



\vspace{2cm}

\textbf{Exercício 02:} 
Seja $M$ uma variedade diferenciável e $ K \subset O\subset M$, onde $K$ é um compacto não vazio e $O$ é aberto. 
Mostre que existe uma função diferenciável $f:M \rightarrow \mathbb{R}$ tal que $ f(x)>0$ se $x \in K$ e $f(x) = 0$, se $ x \notin O$.

\vspace{0.5cm}

\textbf{Resolução:} 
Dado $p \in K$, queremos mostrar que existe uma função suave $g_p : M \rightarrow [0, + \infty)$ tal que $g_p(p) > 0$ e $\text{supp}g_p \subset O$. 
Seja $(\mathcal{U}, \psi : \mathcal{U}\rightarrow \R^n)$ uma carta de $M$ centrada em $p$. 
Como $O$ é um aberto de $M$ e $K \subset O$, temos que $(\mathcal{U} \cap O , \psi : \mathcal{U} \cap O \rightarrow \R^n)$ é uma carta de $M$ centrada em $p$. 
Assim, seja 
\begin{equation*}
B_s(0) = \{x \in \R^n : ||x || < s\}
\end{equation*} 
uma bola aberta de raio $s > 0$ tal que $\psi(p) \in B_s(0)$, o que é claro, pois $\psi(p) = 0$. 
Tome $r > 0$ tal que $0 < r < s$.
Pelo \textbf{Exercício 1}, existe uma função diferenciável $f_p : \R^n \rightarrow \R$ tal que $f(x) = 1$ se $x \in \overline{B}_r(0)$ e $\text{supp}f_p \subset B_s(0)$. 
Defina 
\begin{align*}
g_p : M &\rightarrow [0, + \infty) \\ 
q &\mapsto \begin{cases}
f_p(\psi(q)) & \text{se}\;  q \in \psi^{-1}(B_s(0)) \\
0  & \text{caso contrário}
\end{cases}
\end{align*}

Mostremos que $\text{supp} g_p \subset O$.
Como $\overline{B_r(0)} = B_r[0] \subset B_s(0)$, temos que $\overline{\psi^{-1}(B_r(0))} \subset \psi^{-1}(B_s(0)) \subset O$. 
Note que, $g_p$ é diferenciável em $M\backslash \mathrm{supp}g_p$ pois ela se anula neste subconjunto, e $g_p$ é diferenciável em $\psi^{-1}(B_s(0))$. 
Dado que $(M\backslash \mathrm{supp} g_p) \cup \psi^{-1}(B_s(0))$=M, temos a diferenciabilidade de $g_p$ em $M$.

Seja $U_p \coloneqq \{q \in M : g_p(q) > 0\}$.
Note que $U_p$ é aberto pois $U_p = g_p^{-1}(0,\infty)$.
Logo, $\{U_p\}_{p \in K}$ é uma cobertura aberta para $K$, como $K$ é compacto, existe uma subcobertura aberta $\{U_{p_i}\}_{i = 1}^n$. 
Defina $f : M \rightarrow [0, + \infty)$ por $f(q) = \sum_{i = 1}^n g_{p_i}(q)$.
Claramente, $f$ está bem definida e é diferenciável pois é uma soma finita de funções diferenciáveis. 
E pela forma que $f$ foi definida ela claramente satisfaz as condições do exercício.

\end{document}
