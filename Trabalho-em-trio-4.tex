\documentclass[twoside,openright,titlepage,numbers=noenddot,headinclude,  lineheaders footinclude=true,cleardoublepage=empty,BCOR=5mm,paper=a4,fontsize=12pt ]{scrbook} 
\usepackage[usenames,dvipsnames,svgnames,table]{xcolor}
\usepackage[unicode]{hyperref}
\usepackage[utf8]{inputenc}
\usepackage[eulerchapternumbers,beramono]{classicthesis} 
\usepackage{graphicx}
\usepackage[brazil,portuguese, english]{babel}
\usepackage{hyperref}
\usepackage{mathtools}
\usepackage{relsize}
\usepackage{amsmath,pict2e}
\usepackage{amsfonts}
\usepackage{amssymb}
\usepackage{amsthm}
\usepackage{breqn}
\usepackage{tikz-cd}
\usepackage{pst-node}
\usepackage{pdfpages}
\usepackage{makeidx}
\usepackage{blindtext}
\usepackage[intoc, portuguese]{nomencl}
\usepackage{nomencl}
\usepackage[T1]{fontenc}


\addtolength{\headsep}{0.6cm}
\setlength{\textheight}{22.7cm}
\setlength{\textwidth}{16.2cm}
\setlength{\oddsidemargin}{0cm}
\setlength{\evensidemargin}{0cm}

\linespread{1.3}

\newtheorem{teo}{Teorema}[chapter]
\newtheorem{lema}[teo]{Lema}
\newtheorem{prop}[teo]{Proposição}
\newtheorem{afirm}[teo]{Afirmação}
\newtheorem{cor}[teo]{Corolário}
\newtheorem{defin}[teo]{Definição}
\newtheorem{obs}[teo]{Observação}
\newtheorem{exemplo}[teo]{Exemplo}
\newcommand*{\QEDA}{\hfill\ensuremath{\blacksquare}}
\DeclareMathOperator{\sign}{sgn}
\newcommand{\R}{\mathbb R}
\newcommand{\N}{\mathbb N}
\newcommand{\RN}{\mathbb R^n}
\newcommand{\RM}{\mathbb R^m}
\newcommand{\Q}{\mathbb Q}
\newcommand{\Z}{\mathbb Z}
\newcommand{\K}{\mathbb K}
\newcommand{\mc}{\mathcal}
\newcommand{\ds}{\displaystyle}
\newcommand{\cqd}{\begin{flushright}$_\blacksquare$\end{flushright}}
\newcommand{\ck}{\mc{C}^k}
\newcommand{\g}{\mathfrak g}
\newcommand{\h}{\mathfrak h}
\newcommand{\gl}{\mathfrak {gl}}
\newcommand{\Ad}{\hbox{Ad}}
\newcommand{\oo}{_}
\newcommand{\ad}{\hbox{ad}}
\newcommand{\e}{\mathfrak e}
\newcommand{\uu}{\mathfrak{u}}
\makeatletter
\newcommand{\bigcomp}{%
  \DOTSB
  \mathop{\vphantom{\sum}\mathpalette\bigcomp@\relax}%
  \slimits@
}
\newcommand{\bigcomp@}[2]{%
  \begingroup\m@th
  \sbox\z@{$#1\sum$}%
  \setlength{\unitlength}{0.9\dimexpr\ht\z@+\dp\z@}%
  \vcenter{\hbox{%
    \begin{picture}(1,1)
    \bigcomp@linethickness{#1}
    \put(0.5,0.5){\circle{1}}
    \end{picture}%
  }}%
  \endgroup
}
\newcommand{\bigcomp@linethickness}[1]{%
  \linethickness{%
      \ifx#1\displaystyle 2\fontdimen8\textfont\else
      \ifx#1\textstyle 1.65\fontdimen8\textfont\else
      \ifx#1\scriptstyle 1.65\fontdimen8\scriptfont\else
      1.65\fontdimen8\scriptscriptfont\fi\fi\fi 3
  }%
}

\makeatother
\title{Lista 4 Riemanniana}
\author{}
\date{Outubro 2021}

\begin{document}


\maketitle

\noindent\textbf{Exercício:} Considere $\mathbb{R}_+^2 = \left\lbrace (x, y) \in \mathbb{R}^2 ; \ y > 0 \right\rbrace$ e a variedade Riemanniana $$ M = \left( \mathbb{R}_+^2, \frac{\delta _{ij}}{y^2} \right) .$$
Mostre que se $c : \left[ 0, + \infty \right. ) \rightarrow M$ é uma curva parametrizada divergente, então seu comprimento é infinito, ou seja, $$ \displaystyle \lim _{b \rightarrow + \infty} \ell _0^b (c) = + \infty .$$

\begin{enumerate}
    \item Prove que o comprimento de $\beta : \left[ 0, 1 \right. ) \rightarrow M$, $\beta (t) = (0, 1 - t)$ é infinito. Faça o mesmo para $\eta : \left[ 1, + \infty \right. ) \rightarrow M$, $\eta (t) = (0, t)$.

\noindent\textit{Dem:}  
    $$ \begin{array}{ccl}
\ell _0^1  (\beta)& = & \displaystyle \lim _{a \rightarrow 1^-} \int _0^a \sqrt{\langle \beta '(t), \beta '(t) \rangle} dt \\
& = & \displaystyle \lim _{a \rightarrow 1^-} \int _0^a \sqrt{\langle (0, -1), (0, -1) \rangle} dt \\ 
& = & \displaystyle \lim _{a \rightarrow 1^-} \int _0^a \sqrt{\frac{1}{(1 - t)^2}} dt \\ 
& = & \displaystyle \lim _{a \rightarrow 1^-} \int _0^a \frac{1}{1 - t}dt \\ 
& = & \displaystyle \lim _{a \rightarrow 1^-} (- \ln(\vert 1 - a \vert) - (- \ln (\vert 1 - 0 \vert )) ) \\
& = & \displaystyle \lim _{a \rightarrow 1^-} - \ln (\vert 1 - a \vert) \\
& = & + \infty 

\\

\\ 

\ell _1^{\infty} (\eta) & = & \displaystyle \lim _{b \rightarrow + \infty} \int _1^b \sqrt{\langle \eta '(t), \eta '(t) \rangle} dt \\ 
& = & \displaystyle \lim _{b \rightarrow + \infty} \int _1^b \sqrt{\langle (0, 1), (0, 1) \rangle} dt \\
& = & \displaystyle \lim _{b \rightarrow + \infty} \int _1^b \sqrt{\frac{1}{t^2}}dt \\ 
& = & \displaystyle \lim _{b \rightarrow + \infty} \int _1^b \frac{1}{t} dt \\ 
& = & \displaystyle \lim _{b \rightarrow + \infty} (\ln(b) - \ln(1)) = + \infty .
    \end{array} $$
    
    
    \item Mostre que o comprimento de $c(t) = (c_1(t), c_2(t))$ em $M$ é maior ou igual ao comprimento de sua projeção $(0, c_2(t))$ no eixo $y$. 
    
    \noindent\textit{Dem:} Seja $c:[a,b]\rightarrow M$ definida por $c(t) = (c \oo 1 (t), c \oo 2 (t))$.
    Considere $\pi \oo 2 (c) (t) =(0, c \oo 2 (t)) $ a projeção de $c$ no eixo $y$. 
    Vejamos que $\ell \oo a ^b (c) \geq  \ell \oo a ^b (\pi \oo 2 (c))$. 
    Em verdade, note que
    \[
        \ell \oo a ^b (c) = \int \oo a ^b \dfrac{\sqrt{{c'} ^2 \oo 1 (t) + {c'}^2 \oo 2 (t)}}{c\oo 2 (t)} dt \geq \int \oo a ^b \dfrac{\sqrt{{c'} ^2 \oo 2 (t) }}{c\oo 2 (t)} dt  = \ell  \oo a ^b (\pi \oo 2 (c)).
    \]
    \item Considere o retângulo $\mathcal{Q} = (-A, A) \times (\delta, R) \subset M$, com $\delta, R, A > 0$ e $\delta < R$. Observe que o comprimento de todas as curvas parametrizadas em $\left( \mathcal{Q}, \frac{\delta _{ij}}{y^2} \right)$ é maior do que o comprimento das curvas em $\left( \mathcal{Q}, \frac{\delta _{ij}}{R^2} \right)$. 
    
    \noindent\textit{Dem:} De fato, fixe $R, \delta, A > 0$ em $\mathbb{R}$ de tal forma que $\delta < R$ e considere o conjunto $\mathcal{Q} = (-A,A) \times (\delta,R)$. 
    Em $\mathcal{Q}$, considere as métricas $\left(\frac{\delta_{ij}}{R^2}\right)$ e $\left(\frac{\delta_{ij}}{y^2}\right)$.
    Afirmamos que, para toda curva parametrizada $\gamma: [a,b] \longrightarrow \R^2_{+}$, o comprimento de $\gamma$ em $\left(\mathcal{Q},\frac{\delta_{ij}}{y^2}\right)$ é maior que em $\left(\mathcal{Q},\frac{\delta_{ij}}{R^2}\right)$.
    Com efeito, se $(x,y) \in \left(\mathcal{Q},\frac{\delta_{ij}}{y^2}\right)$, então $R > y.$
    Portanto $\frac{1}{y} > \frac{1}{R}$.
    Considerando $\gamma(t) = (\gamma_1(t),\gamma_2(t)) \in \mathcal{Q}$, então $\gamma_2(t) < R$ para todo $t \in [a,b]$. 
    Se $\ell_a^b$ e $\hat{\ell}_a^{b}$ são os respectivos comprimentos em $\left( \mathcal{Q}, \frac{\delta _{ij}}{y^2} \right)$ e $\left( \mathcal{Q},  \frac{\delta _{ij}}{R^2} \right)$, então 
    \begin{eqnarray*}
        \ell_a^b(\gamma) &=&\int_a^b ||\gamma'(t)||_{\gamma(t)}dt \\
        &=&\int_a^b \sqrt{\frac{\gamma_1'(t)^2 + \gamma_2'(t)^2}{\gamma_2(t)^2}} dt\\
        &\geq& \int_a^b \sqrt{\frac{\gamma_1'(t)^2 + \gamma_2'(t)^2}{R^2}}dt\\
        &=&\hat{\ell}_a^b(\gamma). 
    \end{eqnarray*}
    como gostaríamos.  
    
    
    \item Use os itens de $1$ a $3$ para mostrar que para todo $L>0$, existe um retângulo $\mathcal{Q}_L \subset M$ tal que toda curva parametrizada que começa em $(0,1) \in M$ e sai do retângulo tem comprimento maior do que $L$.    
    
    \noindent\textit{Dem:} Seja $L>0$. Defina
    \[
        \mathcal{Q} \oo L  = \left[-e^{2L+2}, e^{2L+2}\right]
        \times \left[1 - \dfrac{1}{e^{L+1}}, e^{L+1} \right].
    \]
    Seja $\gamma : [a,b) \rightarrow M$ uma curva diferenciável com $\gamma (a) = (0,1)$ e $\gamma([a,b))\not \subset \mathcal{Q}\oo L $.
    Seja $t \oo 0 \in [a,b)$ o menor tempo tal que $\gamma (t \oo 0) \not \in \mathcal Q_L$.
    Neste caso, $\gamma (t\oo 0) \in \partial \mathcal{Q} \oo L$, pela continuidade de $\gamma$. 
    Se $\gamma \oo 2 (t \oo 0) = e^{L+1}$, então $ \eta ([1, e^{L+1}])\subset \pi \oo 2 (\gamma)([a, t \oo 0])$, temos
    \[
        \ell \oo a ^b (\gamma ) \geq \ell  \oo a ^{t \oo 0}(\pi\oo 2(\gamma))\geq \ell  \oo 1 ^{e^{L+1}} ( \eta ) = \int \oo 1 ^{e^{L+1}} \dfrac{1}{t}dt = \ln (e^{L+1}) - \ln (1) = L+1 >L.
    \]
    Se $\gamma \oo 2 (t \oo 0) = 1-\frac{1}{e^{L+1}}$, então $\beta   ([0, 1 - \frac{1}{e^{L+1}} ]) \subset \pi \oo 2 (\gamma) ([a, t\oo 0]) $. Assim
    
    \[
        \begin{array}{rcl}
            \ell \oo a ^b (\gamma ) 
            &\geq & \ell  \oo a ^{t \oo 0}(\pi\oo 2(\gamma))\\
            &\geq &\ell  \oo 0 ^{1-\frac{1}{e^{L+1}}} ( \beta ) \\
            &=&     \displaystyle\int \oo 0 ^{1-\frac{1}{e^{L+1}}} \dfrac{1}{1-t}dt \\
            &=& -\ln \left(1-\left({1-\frac{1}{e^{L+1}}}\right)\right) + \ln (1-0)\\
            &=&  -\ln (e^{-L-1})  \\
            &=&L+1\\
            &>& L.
        \end{array}
    \]
    Considere $r=e^{L+1}$. 
    Suponha $\gamma \oo 1 (t \oo 0)  = e^{2L+2}$.
    O caso $\gamma \oo 1 (t \oo 0)  = -e^{2L+2} $ é análogo.
    Considere a curva $ \xi :[a,t \oo 0]\rightarrow M$ definida por $\xi (t) = (\gamma \oo 1 (t), r)$.
    Assim, $\xi ([a,t \oo 0]) \subset \mathcal{Q}\oo L$ e 
    \[
        \ell \oo a ^{t \oo 0 } (\gamma) =\int_a^{t \oo 0} \sqrt{\frac{\gamma_1'(t)^2 + \gamma_2'(t)^2}{\gamma_2(t)^2}} dt
        \geq 
        \int_a^{t \oo 0} \frac{\sqrt{\gamma_1'(t)^2}}{r} dt
        = \ell \oo a^{t \oo 0} (\xi)
    \]
    Consideremos $\alpha : [0, e^{2L + 2}] \rightarrow M$ dada por $\alpha (t) = (t, r)$. 
    Então $$ \alpha ([0, e^{2L + 2}]) \subset \xi ([a, t_0]) $$ e, portanto, $$ \ell _a^{t \oo 0} (\xi) \geq \ell _0 ^{e ^{2L + 2}} (\alpha) .$$
    Vejamos que $$\begin{array}{ccl} 
    \ell _a ^{t \oo 0} (\alpha) & = & \displaystyle \int _0 ^{e^{2L + 2}} \sqrt{\langle (1, 0), (1, 0) \rangle} dt \\
    & = & \displaystyle \int _0 ^{e^{2L + 2}} \frac{1}{r} dt \\ 
    & = & \frac{1}{r} e^{2L + 2} > e^{L + 1} > L,
    \end{array}$$
    pois $r \leq e^{L + 1}$.
    Logo, $\ell _a^{t \oo 0} (\xi) > L$.
    
    Sendo assim, $\ell _a^b (\gamma) \geq \ell _a^{t_0} (\gamma) > L$.




\item Junte os itens de 1 a 4 e  conclua o exercício.

Seja $c : \left[ 0, +\infty \right ) \longrightarrow M$ uma curva parametrizada divergente, com $c (0) = (0, 1)$. 
Dado $L > 0$, consideremos o retângulo $\mathcal{Q}_L$ do item anterior.
Considerando o compacto $\overline{\mathcal{Q}_L}$, temos que existe $T > 0$ de forma que $c(t) \notin \overline{\mathcal{Q}_L}$ para todo $t > T$.
Em particular, $c$ sai de $\mathcal{Q}_L$ e, pelo item anterior, segue que $\ell (c) > L$.
Sendo assim, $$ \displaystyle \lim _{b \rightarrow \infty} \ell _0^b (c) = + \infty .$$
\end{enumerate}

Para o caso geral, consideremos uma curva parametrizada divergente $c:[0, \infty) \longrightarrow M$ qualquer e denotemos $c(t) = (c_1(t), c_2(t))$.
Então a curva começa em $(c_1(0), c_2(0))$.

Consideremos $f : M \rightarrow M$ dada por $$ f(x, y) = \left( \frac{x - c_1(0)}{c_2(0)} , \frac{y}{c_2(0)}\right) .$$

Vejamos que $$ df_{(p, q)}(\alpha '(0)) = (f \circ \alpha)'(0) = \left( \frac{\alpha _1'(0)}{c_2(0)}, \frac{\alpha _2 '(0)}{c_2(0)} \right) = \frac{\alpha '(0)}{c_2(0)} .$$

Portanto, como $df_{(p, q)}$ é um isomorfismo, segue que $f$ é um difeomorfismo local. Como $f$ é claramente bijetora, segue que $f$ é um difeomorfismo.

Por fim, dado $(p, q) \in M$, temos 
\begin{equation*}
\begin{array}{rcl}
\langle (u_1, v_1), (u_2, v_2) \rangle _{(p, q)} & = & \dfrac{u_1u_2 + v_1v_2}{q^2} \\
& = & \dfrac{\dfrac{u_1u_2}{c_2(0)^2} + \dfrac{v_1v_2}{c_2(0)^2}}{\dfrac{q^2}{c_2(0)^2}} \\
& = & \langle df_{(p, q)}((u_1, v_1)), df_{(p, q)}((u_2, v_2)) \rangle _{f(p, q)}
\end{array}
\end{equation*}
e, portanto, $f$ é uma isometria.

Além disso, $f(c(0)) = (0, 1)$.
Portanto, como o resultado é puramente métrico e válido para curvas que começam em $(0, 1)$, segue que o resultado é válido para a curva $c$.
\end{document}
