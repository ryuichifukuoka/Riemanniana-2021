\documentclass{article}[11pt]

\usepackage[latin1]{inputenc}
\usepackage{amsmath}
\usepackage{amssymb}
\usepackage{color}
\usepackage{hyperref}
\usepackage{stackrel}
\usepackage{ulem}

\newtheorem{theorem}{Theorem}[section]
\newtheorem{lemma}[theorem]{Lemma}
\newtheorem{example}[theorem]{Example}
\newtheorem{definition}[theorem]{Definition}
\newtheorem{conjecture}[theorem]{Conjecture}
\newtheorem{proposition}[theorem]{Proposition}
\newtheorem{remark}[theorem]{Remark}
\newtheorem{corollary}[theorem]{Corollary}
\newtheorem{observation}[theorem]{Observa\c c\~ao}
\newtheorem{exercicio}[theorem]{Exerc\'\i cio}

%\addtolength{\topmargin}{-2cm}
%\addtolength{\textheight}{15cm}

\DeclareMathOperator{\sen}{sen}

\begin{document}

\section{Variedades Riemannianas são espaços métricos}

\

\begin{observation}
Procurou-se seguir as notações do livro de geometria Riemanniana do Manfredo do Carmo.
Lembre-se que diferenciável significa de classe $C^\infty$.
\end{observation}

Seja $(M,g)$ uma variedade Riemanniana conexa.
O comprimento de uma curva diferenciável por partes $c : [a,b] \rightarrow M$ é dada por
\[
\ell(c) = \int_a^b \left< c^\prime(t), c^\prime(t)\right>^{1/2}.dt,
\]
onde $\left< \cdot, \cdot\right>$ denota a métrica Riemanniana $g$. 
Neste caso, dizemos que $c$ liga $c(a)$ a $c(b)$.
Denotaremos a família de caminhos diferenciáveis por partes em $M$ que ligam $p$ a $q$ por $\mathcal A_{p,q}(M)$. 

\begin{exercicio}
\label{reparametrizacao}
Seja $\tilde t:[\tilde a, \tilde b] \rightarrow [a,b]$ um homeomorfismo diferenciável por partes.
Prove que $c\circ \tilde t$ é diferenciável por partes e que $\ell(c) = \ell(c \circ \tilde t)$. 
\end{exercicio}

O homeomorfismo acima pode ser tanto estritamente crescente como estritamente decrescente.

Seja $\iota:[a,b] \rightarrow [a,b]$ a única função afim tal que $\iota(a) = b$ e $\iota(b) = a$.
A reversa de $c \in \mathcal A_{p,q}(M)$ é a curva $\bar c = c \circ \iota \in A_{q,p}(M)$.
Pelo exercício \ref{reparametrizacao}, temos que $\ell_g(c) = \ell_g(\bar c)$.

Se $p,q \in M$, a distância entre $p$ e $q$ é definida por
\[
d(p,q)=\inf_{c\in \mathcal A_{p,q}(M)} \ell(c).
\]
Mostremos que $d$ é uma métrica em $M$, cuja topologia coincide com a topologia original de $M$.

É imediato que $d\geq 0$.

\begin{exercicio}
Use a igualdade $\ell(c) = \ell(\bar c)$ e mostre que $d(p,q) = d(q,p)$ para todo $p,q \in M$.
\end{exercicio}

Sejam $c_1\in \mathcal A_{p_1,p_2}(M)$ e $c_2 \in \mathcal A_{p_2,p_3}(M)$.
Reparametrize $c_1$ e $c_2$ definido-os nos intervalos $[0,1/2]$ e $[1/2,1]$ respectivamente e seja $c_3$ a concatenação de $c_1$ e $c_2$.
Observe que 
\[
\ell(c_3) = \ell(c_1) + \ell(c_2).
\]

\begin{exercicio}
Mostre que $d(p,q) \leq d(p,r) + d(r,q)$ para todo $p,q,r \in M$.
\end{exercicio}

Sejam $p,q\in M$ pontos distintos. 
Para mostrarmos que $d(p,q)>0$, vamos comparar $g$ com uma métrica Euclidiana $h$ em sistema de coordenadas, a qual denotaremos por $(\cdot,\cdot)$.

Seja $\mathbf x: U \rightarrow M$ uma parametrização em $p$ tal que $\mathbf x(0)=p$.
Denote a métrica Euclidiana em $U$ por $h=(h_{ij})=(\delta_{ij})$.
Seja $B = B_h(0,\varepsilon) \subset U$ uma bola Euclidiana em $U$ tal que $\bar B_h(0,\varepsilon) \subset U$ e considere o fibrado tangente unitário 
\[
SB = \{(x,y);x \in B;\sqrt{( y,y)} = 1 \}.
\]
$SB$ tem fecho compacto em $TU$. Com isso
\[
L := \inf_{(x,y) \in SB} \sqrt{\left< y,y\right>} > 0
\]
e
\begin{equation}
\label{d mais fina}
\sqrt{\left< y,y\right>} \geq L \sqrt{\left(y,y \right)}.
\end{equation}
Portanto, todo caminho $c$ que liga $p$ a um ponto fora de $B$ satisfaz $\ell(c) \geq L\varepsilon$.

\begin{exercicio}
Conclua que $d(p,q)>0$ se $p\neq q$. 
Logo $d$ é uma métrica em $M$.
\end{exercicio}

\begin{exercicio}
Mostre que a topologia $\mathcal T_d$ gerada pela métrica $d$ coincide com a topologia original $\mathcal T_M$ de $M$.

Sugestão: Conservando a notação acima, note que $\mathcal T_M$ é gerada por 
\[
\left\{ \mathbf x(B_h(0,\varepsilon)), \mathbf x \text{ parametriza\c c\~ao de M, }\varepsilon > 0 \right\}.
\] 
Mas por (\ref{d mais fina}) sempre existe uma bola $B_d(p,\eta) \subset M$ tal que $B_d(p,\eta) \subset \mathbf x(B_h(0,\varepsilon))$.
Portanto $\mathcal T_M \subset \mathcal T_d$.
Para a outra inclusão, defina
\[
H := \sup_{(x,y) \in SB} \sqrt{\left< y,y\right>} > 0,
\]
e note que $\sqrt{\left< y,y\right>} \leq H \sqrt{( y,y)}$ se $(x,y) \in TB$. 
Conclua que as duas topologias coincidem.

\end{exercicio}

\end{document}
